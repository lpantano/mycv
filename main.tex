\hypertarget{summary}{%
\section{Summary}\label{summary}}

Computational biologist expert in high-dimensional data, from sequencing
data to microscope screening. 13 years of experience in biological data
analysis using the most well-established tools and contributing to novel
algorithms to improve the quantification and visualization of genomic
data. I approach scientific projects with passion and believe that a
team and not an individual alone can successfully conquer them.

\hypertarget{competences}{%
\section{Competences}\label{competences}}

Building Partnerships, Building Trust, Continuous Learning, Decision
Making, Delivers Results, Interpersonal Skills, Planning and Organizing,
Problem Analysis and Problem Solving, Pursues Excellence, Teamwork and
Collaboration, Time Management, Valuing Diversity

\hypertarget{education}{%
\section{Education}\label{education}}

\begin{itemize}
\tightlist
\item \textbf{Coursera} \hfill \textbf{2020}.
  \newline
  \textbf{Certificate} in \textit{Neural Networks and Deep Learning}
\item
  \textbf{Center for Genomic Regulation}, Barcelona, Spain \hfill \textbf{2011}
  \newline
  \textbf{Ph.D.} in \textit{Biomedicine}
  \newline
  \textit{Full characterization of the small RNA transcriptome using novel computational methods for high-throughput
sequencing data: study of  miRNA variability in eukaryote organisms}
\item
  \textbf{University of Pompeu Fabra}, Barcelona, Spain \hfill \textbf{2008}
  \newline
  \textbf{M.S.} in \textit{Bioinformatics for health science}
\item
  \textbf{University of Granada}, Granada, Spain \hfill \textbf{2005}
  \newline
  \textbf{B.S} \textit{Biochemistry}
\end{itemize}

\hypertarget{experience}{%
\section{Experience}\label{experience}}

\begin{itemize}

\item
\textbf{Axcella Health}, Cambridge, MA \hfill \textbf{2020 - to date}
\newline
\textbf{Principal Scientist in Computational Biology}
\begin{itemize}
\item
  Leading R\&D pipelines with computational strategies
\item
  Integrating genomic/metabolimic/transcriptomic data to understand gene regulation
\item 
  Leading visualization plarform to boost biological interpretation
\item 
  Developing bioinformatic infrastructure to ensure reproducibility and scalability in the cloud
\item
  Experience with CROs
\item
  Senior Manager and Mentor
\end{itemize}

\item
\textbf{eGenesis}, Cambridge, MA \hfill \textbf{2019 - 2020}
\newline
\textbf{Senior Computational Biologist}
 \begin{itemize}

\item
  Responsable of the internal next generation sequencing team with oxford nanopore and illumina sequencers
\item
  Research of putative chromatine opening genomic elements to enforce expression of transgenes through machine learning methods
\item
  Technical lead in the quality control of modified pigs from genomic data
\item 
  Develop small RNAseq, (Single Cell and bulk) RNA-seq, and ChIP-seq pipelines
  as part of the nf-core community
\item System Architect for the computational infrastructure based on the cloud
\item
  Help with manuscript materials
\item Mentor junior researchers in the group
 \end{itemize}


\item
\textbf{Picower Institute of MIT}, Cambridge, MA \hfill \textbf{2019}
\newline
\textbf{Research Scientist - Bioinformatics Core Supervisor}
 \begin{itemize}
\item
  Organize and maintain bioinformatics pipelines for researchers
\item
  Help with grant and manuscript materials
\item 
  Develop small RNAseq, (Single Cell and bulk) RNA-seq, and ChIP-seq pipelines
  as part of the nf-core community
\item
  Develop variant calling pipeline as part of the bcbio-nextgen
\item 
  Develop supervised and unsupervised clustering, differential expression, and visualization analyses
  in python frameworks and the Bioconductor platform
\item Mentor junior researchers in the group
 \end{itemize}

\item
  \textbf{Boehringer Ingelheim}, \hfill \textbf{2016 - 2019}
  \newline
  \textbf{Computational Biologist - Fellowship, 80\%}
 \begin{itemize}
  \item  Lead the bioinformatic analysis (at Harvard) in a international collaboration for the study of fibrosis across organs 
  \item Design and analyze the RNAseq and smallRNAseq data in a pre-clinical NASH study
  \item Design, automatize and optimize the high-throughput screening pipeline of small molecules in three cell models
  \item Manage junior bioinformatician
\end{itemize}

\item
  \textbf{Harvard T.H. Chan School of Public Health}, Boston, MA \hfill \textbf{2017 - 2019}
  \newline
  \textbf{Research Scientist}
 \begin{itemize}
  \item
    Organize the data science research in the small RNA-seq field inside the  (\href{https://mirtop.github.com}{\textbf{miRTOP}}) group
  \item
    Lead the bioinformatic analysis in the cross organ fibrosis project
    collaboration between Harvard University and Boehringer Ingelheim
  \item
    Develop tools for NGS data analysis: small RNAseq, (Single
    Cell and bulk) RNA-seq, ChIP-seq, WholeGenome-seq as part of the bcbio-nextgen
    python framework and Bioconductor platform
  \item Mentor junior researchers in the group
\end{itemize}

\item
  \textbf{Harvard T.H. Chan School of Public Health}, Boston, MA \hfill \textbf{2014-2017}
  \newline
  \textbf{Research Associate}

  \begin{itemize}
  \tightlist
  \item Developed the visualization and integration of multi-omics data
  % TODO add web page for the app tool
  \item
    Developed pipelines for NGS data analysis (ATACseq, ChIPseq, RNAseq) inside the bcbio-nextgen python framework
  \item
    Developed co-correlation methods to detect transcriptional modules from transcriptome data
  \end{itemize}
\item
  \textbf{Institute of Biotechnology and Biomedicine}, Barcelona, Spain \hfill \textbf{2013 - 2014}
  \newline
  \textbf{Post-doctoral fellow}

  \begin{itemize}
  \tightlist
  \item
    Detected de-novo transcripts as effect of inversion in the HapMap population
  \item
    Expression quantitive trait loci analysis of inversion in the HapMap population
    % TODO fix sentence
  \item
    Developed inversion database back-end
  \end{itemize}
\item
  \textbf{ASCIDEA}, Barcelona, Spain \hfill \textbf{2011-2014}
  \newline
  \textbf{Co-founder and CTO}

  \begin{itemize}
  \tightlist
  \item
    Led a group of 3 bioinformaticians
  \item
    Supervised cloud computing systems
  \item
    Developed computational pipelines
  \end{itemize}
\item
  \textbf{Institute of Predictive and Personalized Medicine of Cancer}, Barcelona, Sapin \hfill \textbf{2011--2013}
  \newline
  \textbf{Post-doctoral fellow}

  \begin{itemize}
  \tightlist
  \item
    Determined small RNA characterization in human sperm samples
  \end{itemize}
\end{itemize}

\hypertarget{tools}{
\section{Bioinformatics tools}\label{tools}}

\begin{itemize}
\tightlist
\item
  \href{http://github.com/lpantano/seqcluster}{\textbf{seqcluster}}:
  python package for smallRNA detection and quantification
\item
  \href{http://github.com/lpantano/seqclusterViz}{\textbf{seqclusterViz}}:
  Web-app for smallRNAseq visualization
\item
  \href{http://github.com/lpantano/isomiRs}{\textbf{isomiRs}}:
  R/Bioconductor package for isomiR detection and quantification
\item
  \href{http://github.com/lpantano/DEGreport}{\textbf{DEGreport}}:
  R/Bioconductor package for Differential Expressed results analysis
  and visualization
\item 
  \href{https://nf-co.re/}{nf-core}: community driven pipelines
\item
  \href{http://github.com/chapmanb/bcbio-nextgen}{bcbio-nextgen}:
  python framework for the analysis of sequencing data
\item
  \href{http://github.com/chapmanb/cloudbiolinux}{cloudbiolinux}:
  python framework for the installation of specific tools related to
  a given bioinformatic analysis
\item
  \href{https://github.com/bioconda/bioconda-recipes}{bioconda}:
  package manager for the installation of bioinformatics tools
\item
  \href{https://github.com/ewels/MultiQC}{multiqc}: python package
  for the generation of reports
\end{itemize}

\hypertarget{instructor-activities}{%
\section{Instructor activities}\label{instructor-activities}}

\begin{itemize}
\tightlist
\item
  Small RNAseq analysis, Harvard T.H. Chan School of Public Health, Boston, MA \hfill 2015-2018
\item
  Data visualization, PRBB, Spain \hfill 2013
\item
  Workshop of ChIP-Seq analysis, University of Vic, Spain \hfill 2013
\item
  Workshop of Metagenomic, University of Vic, Spain \hfill 2013
\item
  Workshop of small RNA analysis, CRG, Spain \hfill 2010
\end{itemize}

\hypertarget{reviewer-experience}{%
\section{Reviewer experience}\label{reviewer-experience}}

\begin{itemize}
\tightlist
\item Editor at JOSS \hfill 2019-todate
\item
  RNA journal \hfill 2018
\item
  NAR journal \hfill 2017-2018
\item
  Scientific Reports journal \hfill 2018
\item
  Bioinformatics journal \hfill 2018-2019
\item
  BMC Genomics journal \hfill 2017
\item
  BOSC program committee \hfill 2014-2018
\item
  Student Council Symposium program committee \hfill 2010-2012
\end{itemize}

\hypertarget{leadership-activities}{%
\section{Leadership activities}\label{leadership-activities}}

\begin{itemize}
\tightlist
\item
  Program committee, Bioconductor Conference \hfill 2019
\item
  Founder and Co-organizer of
  \href{https://www.meetup.com/Cambridge-woman-developers-in-bioinformatics/}{women
  in bioinformatics} MeetUp, Cambridge, MA \hfill 2014-to date
\item
  Program committee, Bioinformatics Open Source Conference \hfill 2014-to date
\item
  Co-founder of \href{http://www.rsgspain.org}{RSG Spain}, the Spanish
  student group of ISCB student council \hfill 2010
\item
  Former member of \href{http://www.iscbsc.org}{ISCB student council} \hfill 2010-2014
%\item
%  Master Thesis Supervisor, Universidad Autonoma de Barcelona \hfill 2014
\end{itemize}

\hypertarget{awards-honors}{%
\section{Awards and Honors}\label{awards-honors}}

\begin{itemize}
\tightlist
\item Leadership Award (eGenesis) \hfill 2020
\item Research Scientist Award (Harvard Chan School) \hfill 2018
\item
  Travel Fellowship for Student Council Organization \hfill 2011
\item
  Travel Fellowship for ISMB participation \hfill 2010
\item
  Training University Lecturers fellowship \hfill 2008-2011
\item
  University Schoolarship \hfill 2000-2005
\end{itemize}


\hypertarget{work-presented-at-international-conferences}{%
\section{Work presented at International
Conferences}\label{work-presented-at-international-conferences}}

\begin{itemize}
\tightlist
\item
  Great Lakes Bioinformatics Consortioum \hfill 2017-Chicago
\item
  Bioinformatics Open Source Conference \hfill 2018-Orlando, 2016-Florida
\item
  CodeFest Hacklaton \hfill 2016-Florida, 2010-Boston
\item
  Bioconductor Annual Conference \hfill 2015-Boston, 2017-Boston
\item
  Intelligent System for Molecula Biology Annual Conference \hfill 2011-Vienna, 2010-Boston
\end{itemize}


\hypertarget{event-organization}{%
\section{Event Organization}\label{event-organization}}

\begin{itemize}
\tightlist
\item
  Mind the Gap Barcelona, Co-Chair \hfill 2013
\item
  ISCB Student Council Symposium Long Beach, Travel fellowship chair  \hfill 2012
\item
  Spain, Portugal and North Africa Student Symposium Barcelona, Program committee chair \hfill 2012

\item
  ISCB Student Council Symposium Vienna, Co-Chair \hfill 2012
\item
  Spain, Portugal and North Africa Student Symposium Malaga, Co-Chair \hfill 2011
\end{itemize}

\hypertarget{skills}{%
\section{Computational skills}\label{skills}}
\begin{itemize}
\tightlist
  \item Programming languages: Python (bioconda packages), R (Bioconductor), Java, Bash, LaTeX
  \item Visualization: HTML, CSS, javascript (D3, Charts, jquery)
  \item Databases: MySQL, sqlite3
  \item AWS cloud: EC2, S3, RDS, dynamoDB, SQS, Batch
  \item Control version system: git, svn
  \item High performance computing: Slurm, LSF, SGE
\end{itemize}

\hypertarget{courses}{%
\section{Courses}\label{courses}}

\begin{itemize}
\tightlist
\item Improving Deep Neural Networks: Hyperparameter tuning, Regularization and Optimization @ Coursera 2020
\item Neural Networks and Deep Learning @ Coursera 2020
\item Structuring Machine Learning Projects @ Coursera 2020
\item Neural Networks and Deep Learning @ Coursera 2020
\item Materials Characterization and Analysis for Scientists and Engineers @ Harvard Extension
School 2018
\item Initiating and Planning Projects @ Coursera 2018
\item
  Fundamentals and Applications of Microfluidics @ Harvard Extension
  School 2018
\item
  Tissue Engineering for Clinical Applications @ Harvard Extension
  School 2017
\item
  Principles of fMRI 1 @ Coursera 2016
\item
  Practical Machine Learning @ Coursera 2016
\item
  Calculus One @ Coursera 2015
\item
  Regression Model @ Coursera 2014
% \item
%  HPC programming (not finished) @ Coursera 2013
\item
  Analysis of high-throughput sequencing data @ EMBL-EBI 2012
\item
  Web applications @ Udacity (CS253) 2012
\item
  Machine learning @ Coursera 2012
\item
  Introduction to Artificial Intelligence (www.ai-class.org) 2011
\item
  Advance R @ CRG 2008
\end{itemize}
